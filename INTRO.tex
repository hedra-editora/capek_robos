\SVN $Id: INTRO.tex 10893 2012-05-16 19:14:47Z oliveira $
\chapter[Introdução, por Aleksandar Jovanović]{introdução}
\hedramarkboth{introdução}{aleksandar jovanović}

\textsc{Os primórdios} da literatura tcheca remontam ao século~\textsc{x} da era vulgar,
quando as lendas de São Venceslau foram redigidas em eslavo eclesiástico,
idioma que representa o primeiro registro de uma língua eslava e que se tornou
o veículo litúrgico da ortodoxia entre os eslavos. Até por volta do século~\textsc{xv},
crônicas em latim, hinos, romances em prosa e textos sobre histórias de
cavalaria constituíam o cerne da atividade literária. Costuma"-se atribuir a
Tomáš Št\'{i}tný [pronuncia"-se Tomách Chtitni] (\textit{c.}~1331--1401) o papel
de primeiro escritor de importância em terras tchecas. Com efeito, ocorreu um
florescimento da literatura tcheca no apagar das luzes da Baixa Idade Média. A
reforma linguística legada por Jan Hus ou Jan Husinecký [pronuncia"-se Yan
Khússinetski] (1371--1415) legou às gerações seguintes do Renascimento um
idioma bem estruturado, mas a Guerra dos Trinta Anos (1618--1648) resultou em
opressão política que a dinastia dos Habsburgos se encarregaria de comandar.
Com isso, o renascimento efetivo da língua tcheca como veículo literário
surgiria apenas no final do século~\textsc{xviii} e início do \textsc{xix}, com as figuras do
filólogo Josef Dobrovský (1752--1829), do poeta romântico Karel Hinek Mácha
[pronuncia"-se Mákha] (1810--1836) e da romancista Božena N\u{e}mcová
[pronuncia"-se Bójena Niémtsova] (1820--1862). No final do século~\textsc{xix},
seguramente Jan Neruda (1834--1891), ensaísta e poeta, emergiu como personagens
de destaque e, no princípio do século passado, os poetas Petr Bezruč
[pronuncia"-se Bezrutch] (1867--1958) e Otakar Březina [pronuncia"-se Bjezina]
(1868--1929). É nesse contexto que aparece o autor do presente livro.

\section{Karel Tchápek e o inusitado\break a serviço da Humanidade}

Karel Čapek (pronuncia"-se Tchápek)\footnote{ Por comodidade, optamos pela transcrição fonética 
na grafia do nome ao longo desta edição. [N.~da E.]} 
nasceu em 9 de janeiro de 1890 em Malé
Svatoňovice (pronuncia"-se Svátonyovitse), então Austro"-Hungria, hoje República
Tcheca, e morreu em Praga, no dia 25 dezembro de 1938. Era filho do médico
Antonín Tchápek. Tinha dois irmãos: Josef (1887--1945) e Helena (1886--1969). Todos
possuíam talento artístico: Josef,  coautor de diversos textos de Karel, foi
desenhista, ilustrador e pintor cubista; Helena também escrevia. Deixou um
livro de memórias dedicado aos irmãos, sob o título \textit{Meus queridos irmãos}. Josef
chegou a trabalhar com Karel no \textit{Národní listy} (Jornal Popular) e compartilhou o
gosto pela jardinagem, habilidade herdada do pai e que, em 1929, resultou no
livro \textit{Zahradnikův rok} (O ano do jardineiro), escrito por Karel. Nosso
personagem casou"-se, em 1935, com a atriz Olga Scheinpflugová, a quem já
conhecia havia uns quinze anos. Após a morte do marido, Olga escreveu uma obra
quase autobiográfica, intitulada \textit{Český román} (O romance tcheco).

Apenas três meses depois que o regime nazista exigiu a anexação dos Sudetos
à Alemanha, Karel morreu de pneumonia. Menos sorte teve Josef, pois, quando as
tropas nazistas invadiram a Tchecoslováquia, em março de 1939, a residência dos
irmãos Tchápek foi um dos primeiros alvos da polícia política (ambos os irmãos
combatiam, abertamente, o nazismo e qualquer forma de totalitarismo, tendo sido
formalmente declarados inimigos públicos de Berlim) e, assim,  o
escritor e pintor Josef acabou no campo de concentração de Bergen"-Belsen, de onde
nunca retornou. 

Karel saiu jovem de sua cidade natal, situada ao norte da Boêmia. Aos onze
anos, foi enviado ao ginásio em Hradec Králové [pronuncia"-se Khrádets Králove),
onde começou a escrever os primeiros textos. Em abril de 1904, na cidade de
Brno, apareceu seu primeiro texto impresso no jornal semanal chamado \textit{Ned\u{e}le} (Domingo): 
dois poemas intitulados ``\textit{Prosté motivy}'' (Motivos simples). Depois,
foi para Praga, estudou filosofia e estética e começou a colaborar, de modo
muito ativo, nos diários mais influentes da capital tcheca com artigos sobre
literatura e arte. Também teve uma passagem acadêmica na França e Alemanha,
onde se embrenhou em estudos referentes à cultura germânica. Tornou"-se logo
autor teatral proeminente. Homem de pensamento livre, de bons costumes,
tornou"-se o máximo representante da cultura democrática de seu país, advertindo
os compatriotas e o mundo a respeito do perigo dos fundamentalismos
ideológicos, que varreriam a democracia e a cultura humanística tanto do Velho
Continente quanto de qualquer outro ponto no mapa"-múndi. 

Não se deve olvidar o fato de que a Primeira Guerra Mundial exerceu forte
influência sobre a geração a que o escritor pertenceu. O conflito fez impérios
explodirem, matou milhões de pessoas nos mais diversos países do planeta,
redesenhou mapas políticos no mundo (a própria Tchecoslováquia surgiria, como
país independente, em consequência do esfacelamento do Império Austro"-Húngaro)
e sinalizou, de modo claro, através do emprego da tecnologia (aviões, gases
venenosos, armas de repetição, tanques etc.), que a ciência
moderna poderia ser (e foi) empregada para ceifar vidas humanas, de forma
impiedosa, bestial. 

O humor sutilmente corrosivo de muitos dos textos de Tchápek pode ser
associado aos escritos de Jaroslav Hašek [pronuncia"-se  Yároslav Kháchek]
(1883--1923), autor do romance \textit{Osudy dobrého
vojáka Švejka za světové války} (O destino do bom soldado Chveik), 
que também se tornou universalmente conhecido.
Por outro lado, ambos podem ser considerados fundadores de uma linha de culto
ao absurdo na literatura tcheca, cujos sucessores, entre outros, são Vladimir
Páral (1932--) e Bohumil Hrábal (1914--1997), todos eles autores de uma prosa
grotesca, porém eloquente, marchetada de modo cuidadoso e que põe em relevo as
distorções resultantes da burocratização e da alienação. No cinema tcheco, Jiři
Menzel (1938-- ) encarna a linha de culto ao absurdo e grotesco, de modo
inteligente e original, fato que sinaliza um vínculo entre esses escritores e o
cineasta quanto ao modo de conceber a representação da realidade. 

É impossível imaginar o rico cenário cultural tcheco e eslovaco do
entreguerras sem pensar nos irmãos Tchápek, tal foi a importância de ambos. Karel
e o irmão conviviam, de modo constante, com o também autor teatral e diretor de
cinema Vladislav Vančura [pronuncia"-se Vántchura] (1891--1942) e o escritor,
humorista e jornalista Karel Poláček (1892--1945), figuras de relevo no mundo
intelectual da época.

Convém salientar que foram contemporâneos de Vítězslav Nezval (1900--1958),
poeta que desempenhou papel importante no desenvolvimento da vanguarda poética		\EP[-1]
e artística de seu país, foi um dos animadores do movimento denominado poetismo
 e também do surrealismo, tendo se vinculado a figuras internacionais como Paul
Éluard e André Breton.~Também, nesse período, começou a produzir as suas obras
o poeta Jaroslav Seifert (1901--1986), prêmio Nobel de literatura, e o poeta
František Halas (1901--1949), ensaísta e tradutor, um dos líricos mais
importantes da literatura tcheca do século \textsc{xx}, que também merece ser lembrado
como figura de proa da época. Tampouco deve"-se esquecer que Tchápek começou a
escrever poucos anos antes do desaparecimento de Franz Kafka (1883-1924),
escritor tcheco que produziu sua obra em alemão e cuja influência foi
fundamental.

O período inicial da carreira literária de Karel Tchápek situa"-se por volta de
1910, quando ainda cursava filosofia. Seu primeiro livro,
coleção de breves contos, intitulado \textit{Boží muka} (O suplício de Deus) viu a luz
em 1917 e, no mesmo ano, foi publicada a coleção de contos \textit{Trapné povídky} (Histórias aflitivas). 
 Ambos os livros expressavam ansiedades e incertezas e
insinuavam a existência de mistérios que os homens talvez jamais fossem capazes
de desvendar. Nesse período, escreve mais duas obras: \textit{Loupežník} (Salteador) e
uma coletânea de traduções intitulada \textit{Francouzká poezie nove doby} (Poesia francesa contemporânea). 
A influência exercida pelo Pragmatismo --- escola de filosofia
de origem norte"-americana, marcada pela descrença no fatalismo e pela certeza
de que só a ação humana, movida pela inteligência e pela energia, pode alterar
os limites da condição humana --- resultou, ainda em 1918, no volume intitulado 
\textit{Pragmatismus čili filosofie praktického života} (Pragmatismo ou a filosofia da vida prática). 

No princípio da década de 1920, os irmãos Tchápek deixaram o jornal em que
trabalhavam e, assim, a atividade literária de Karel pareceu tomar maior
ímpeto. Começa a produzir obras voltadas para temas utópicos e distópicos:  \textit{A fábrica de robôs}, 
escrito em 1920, logo acaba sendo traduzido para o inglês em 1923, fato que impele o rápido
reconhecimento internacional do autor. Também são desse período \textit{V\u{e}c Makropulos} (O caso
Makropulos), \textit{Továrna na absolutno} (A fábrica do absoluto)
e o romance \textit{Krakatit} (o título da obra evoca o verbo \textit{krákat}, grasnar). Em
1929, o autor escreve uma coleção de histórias rotuladas como \textit{Povídky z jedné a z druhé kapsy} (Contos de um e
outro bolsos), histórias escritas em linguagem
coloquial. Datam do começo dos anos 30, suas colunas publicadas em jornal e que
se dedicam a abordar a vida cotidiana. Também publica uma história infantil \textit{Dašenka čili život št\u{e}n\u{e}te}
(Dáchenka ou a vida de uma cadelinha), e seus diários de viagens, escritos há quase oito décadas, foram capazes de tingir com
tintas multicoloridas a imagem de países como a Grã"-Bretanha, Itália, Espanha
e Holanda. Entre 1928 e 1935, publicou
três volumes resultantes de suas conversas com o primeiro presidente da
Tchecoslováquia, Tomáš G.~Masaryk (1850--1937). Tchápek teve amizade bastante
íntima com Masaryk, figura com quem conviveu bastante e que teve papel
fundamental na criação do estado tchecoeslovaco.

Karel Tchápek notabilizou"-se por escrever com humor, inteligência, a respeito
de grande diversidade de temas, e também por utilizar a língua tcheca com
elevado grau de maestria. Muito antes de a ficção científica ter sido
reconhecida como gênero literário independente, abordou questões seminais
referentes à evolução do ser humano sobre a face da terra. Muitas de suas obras
discutem os aspectos éticos das invenções que marcaram o século passado e, além
de colocar em relevo assuntos como a produção de armas nucleares ou modos de
inteligência pós"-humana, expressou considerável receio em relação a desastres
porvindouros, na Europa e no mundo, tais como violência e poderes ilimitados
das grandes corporações, regimes tirânicos, tentando vislumbrar meios para
salvar a humanidade da autodestruição.  Sob alguns aspectos, ele e os
escritores ingleses Aldous Huxley (1894--1963) e George Orwell (1903--1950),
também autores de obras de ficção fundamentais na história literária do século~\textsc{xx},
parecem compartilhar temores similares com a possibilidade de a liberdade
individual ser esmagada por Estados autoritários e/ou totalitários. O
dramaturgo irlandês George Bernard Shaw (1856--1950) expressava bastante
admiração pelas obras de seu colega tcheco e foi um dos responsáveis pela forte
repercussão que a peça \textit{A fábrica de robôs} teve na intelectualidade europeia.

\textit{R.U.R. (Rosumoví univerzalní roboti)}, ou seja,  ``Robôs Universais Rossum''  ---
traduzido para o português, neste texto, como \textit{A fábrica de robôs} --- tem, no
original,  um título que joga com as assonâncias das palavras: ``Rossum'',
transformado em nome de família lembra, em tcheco, o substantivo masculino
\textit{rozum}, ou seja, razão, intelecto, entendimento, ao passo que a palavra \textit{robot}
(cuja invenção o escritor atribuiu ao irmão Josef e que ingressou no universo
lexical de quase todas as línguas contemporâneas) tem ligação etimológica com a
raiz do eslavo eclesiástico \textit{rob} (\russ{робъ}), ``escravo'', e, em tcheco,
com o substantivo feminino \textit{robota}, ``trabalho forçado'' ou ``trabalho físico
extenuante'', e com o verbo \textit{robotit}, ``matar"-se trabalhando''. Em várias outras
línguas eslavas, o universo morfofonológico e semântico desses termos é
similar: em russo, búlgaro, sérvio e macedônio, \textit{rabota} (\russ{рaбота})
quer dizer ``trabalho'' ou ``trabalho físico'', ``faina''; em polonês e eslovaco, \textit{robota}
quer dizer ``trabalho'' ou ``trabalho físico''. ``Robô'', termo que se universalizou, não
tem no texto de Tchápek apenas o sentido de autômato de aspecto humano; o
significado é mais amplo e próximo do de andróide, ou ser humano artificial,
não natural. 

Há uma longa linha, sinuosa, de antecessores históricos do tema distópico
(ou da antiutopia) e dos seres inanimados que se tornam animados. Primeiro,
cabe lembrar a lenda grega de Prometeu, o titã que roubou o fogo dos deuses,
presenteou"-o aos homens --- feitos de barro ---,  de modo a torná"-los superiores às
outras espécies vivas. Os deuses do Olimpo grego condenaram Prometeu a ficar
acorrentado durante 30 mil anos, enquanto um abutre lhe devoraria de modo
incessante o fígado. Talvez não por acaso o substantivo grego \textit{prometh\=es}
signifique antevisão. Depois, e mais ainda, o tema está
vinculado à lenda judaica do Golem, o ser animado feito de material inanimado
que, produzido pelo homem para defendê"-lo de outros homens (dos ataques
antissemitas, mais exatamente), acaba se tornando mau e incontrolável a ponto
de precisar ser destruído. Ao rabino Judah Loew (1525--1609), de Praga, é
atribuída a fixação do primeiro texto referente ao Golem, ao passo que o prêmio
Nobel de Literatura Isaac Bashevis Singer (1902--1991) publicaria, em 1969, a
sua própria versão. Séculos depois, a escritora britânica Mary Shelley
(1797--1851) lançou em 1818 o romance \textit{Frankenstein ou o moderno Prometeu}, texto
que aborda a criação artificial de um ser humano e cujas ações fogem ao
controle do criador. Jack London (1876--1916) publica \textit{The Iron Heel} (O Tacão de
Ferro) em 1908 --- antevisão de uma ditadura totalitária de direita nos Estados
Unidos da América. Há outros textos que enveredam por raciocínio similar e
abordam o perigo de o controle dos atos humanos fugir à espécie a que
pertencemos: Herbert George Wells (1866--1946) publica, em 1895, \textit{A máquina do tempo} 
e o russo Ievguêni Zamiátin (1884--1937) redige um
romance --- \textit{My} (Nós) ---, cujo tema central é o fato de as pessoas viverem no
futuro sob um governo autoritário que controla a vida de todos.

Um dos temas fundamentais que o autor aborda é o temor do mau uso da ciência
e da tecnologia, cuja primeira vítima  seria o homem comum. As graves sombras da
Primeira Guerra Mundial, encerrada em 1918, ainda anuviavam o horizonte, quando
Karel começava a beirar a idade de 30 anos. No que concerne aos riscos de
deturpação das conquistas científicas, não por acaso, nove décadas após a
publicação de \textit{R.U.R}.  um dos debates éticos mais importantes está relacionado,
na atualidade, com a clonagem de seres vivos, fato que, de certo modo, a ficção
de Tchápek deixou antever, embora sua época fosse marcada ainda pelo taylorismo,
ou seja, o máximo de produção e rendimento com o mínimo de tempo e de esforço e
pela desmesurada apologia da técnica, do “progresso” e das máquinas. Ao mesmo
tempo, suas obras permitem entrever o temor de que complexos industriais
colossais fossem capazes de ameaçar a identidade humana ou que exércitos de
robôs insensíveis ou insetos assustadores adquirissem traços humanoides e
apagassem, na prática, as fronteiras entre realidade e ficção.

\textit{A fábrica de robôs} (aqui, numa tradução escorreita, diretamente do original
em tcheco), trata de um tema pouco comum à época em que foi escrito, no ano de
1920. A peça, em três atos, encenada em 1921 no Teatro Nacional de Praga,
discorre a respeito de seres artificiais, trabalhadores incansáveis e
infalíveis, desprovidos de todas as “qualidades desnecessárias” que marcam os
seres humanos, ou seja, não possuem criatividade alguma, não sentem dor nem
possuem qualquer espécie de sentimentos. Nessa sociedade, imaginada por Tchápek,
os robôs acabam assumindo todos e quaisquer encargos humanos, de modo a
racionalizar por completo o processo de produção. Enquanto se atingem níveis
máximos de produtividade, a vida humana torna"-se banal, monótona, quase sem
horizontes, e os homens submergem no gigantesco complexo técnico"-industrial
como ingrediente praticamente sem importância. Com isto, o autor propõe a
seguinte reflexão: Que benefícios para a humanidade poderiam resultar de um
invento revolucionário como esse? 

Uma das respostas está no próprio texto, sob a forma de representação
alegórica: a racionalização absoluta e a desumanização podem conduzir somente à
revolta, à libertação dos grilhões e à aniquilação dos opressores. Os robôs
revoltam"-se na peça de Tchápek, destroem o sistema. Os humanos, peças
minimalistas e desprovidas de importância, encravados no interior de uma
sociedade insensível, nem mesmo são mais capazes de perpetuar a própria
espécie. Por isso, são os robôs que assumem a linha de frente e acabam
extinguindo não somente a sociedade que os criou, mas também a espécie –
hipoteticamente racional e superior às demais espécies --- que os produziu. Novo
pensamento profundo do autor, que sinaliza o fato de que grandes perigos podem
estar mascarados sob a imagem de fórmulas miraculosas, visões grandiloquentes,
que objetivam oferecer à humanidade prosperidade, redenção de qualquer espécie
e boa fortuna. Tanto o stalinismo quanto o nazismo ainda estavam sendo gerados
no ano em que a peça foi redigida, mas, sem sombra de dúvida, o texto
constituiu um alerta contra os fundamentalismos ideológicos que, logo mais, se
abateriam sobre o mundo e que se multiplicariam ainda, anos a fio, nos mais
remotos rincões do globo terrestre.

Um dos personagens de \textit{A fábrica de robôs}, Domin, diretor da fábrica,
vaticina que, no período de uma década, sua unidade produziria volume tão
significante de produtos, que eles deixariam de possuir qualquer valor de troca
e que, nesse momento, os homens simplesmente poderiam recolher tanto quanto
desejassem ou necessitassem. Por conseguinte, as pessoas poderiam passar a vida
em constante e eterna fruição, sem precisar dar atenção a pequenas, mas
incessantes preocupações com o cotidiano. Alquist, herói da peça que defende
valores humanistas básicos e parece temer promessas utópicas, profere uma longa
prece, cujo final reza: “Protege a espécie humana da destruição\ldots{}” Tema 
singularmente atual, quando há dirigentes de nações que se batem, de todas as
formas, com o intuito de dotar suas nações de bombas nucleares, cuja potência e
efeito devastador seriam capazes de varrer qualquer forma de vida do planeta.
Tchápek contrapõe o velho Rossum, fundador da grande fábrica de robôs, e o homem
artificial, indicando que haveria uma linha de continuidade entre o
materialismo científico do século \textsc{xix} e os seres insensíveis, não humanos, fato
que resultaria na superfluidade de o homem crer na existência de uma
inteligência superior no universo, visto que ele próprio, ser humano, teria
assumido o controle sobre todas as coisas.

De modo significativo, o autor confere ao tempo dramático um tratamento
pouco comum:  intitula \textit{A fábrica de robôs} de “drama coletivo”, cujos
personagens acabam sendo simplesmente aniquilados. A extinção da espécie,
contudo, não resulta da ação e/ou vontade de um \textit{deus ex machina} (no teatro
grego da Antiguidade, uma inesperada, artificial ou improvável personagem,
artefato ou evento introduzido de modo repentino para resolver uma situação ou
desembaraçar uma trama), mas na revolta do homem contra as leis da natureza,
que pretende dominar e submeter à sua própria minúscula vontade. 

Em 1923, publica \textit{V\u{e}c Makropulos} (O caso Makropulos) que, três anos
depois, seria encenada na cidade de Brno como ópera, composta por Leoš Janáček
[pronuncia"-se Léoch Yánatchek] (1854-1928) e cujo libreto o compositor baseou
na peça de Tchápek. O texto satiriza a infindável busca humana pela imortalidade
e pela fortuna. Narra a história de Hieronymus Makropulos, médico da corte do
imperador Rudolph \textsc{ii} de Habsburgo que, em 1565, descobre o elixir da
longevidade. O imperador, sem dar crédito ao invento, obriga a filha, Elina, a
beber a poção. Makropulos morre preso, mas a princesa realmente passa a ter
vida quase imortal e, a cada seis ou sete décadas, troca de identidade,
conservando, contudo, as iniciais E.M. No começo do século \textsc{xix}, quando
encarnava a cantora escocesa Ellen MacGregor, teve uma aventura amorosa em
Praga como o barão Prus, de quem gerou um filho, Ferdinand MacGregor. O barão
morre e a herança vai para um primo. Ferdinand aparece como novo pretendente à
fortuna e a pendência judicial entre as famílias Prus e MacGregor vai durar
mais de um século. 

Entre 1933 e 1934, publica três romances --- \textit{Hordubal}, \textit{Pov\u{e}troň} (Meteoro)
e \textit{Obyčejní život} (Uma vida comum) --- que lidam com aspectos particulares da vida
humana. Deixou uma obra inacabada --- \textit{A vida do compositor Foltýn}.

Em 1936, publicou \textit{A guerra das salamandras},\footnote{ Publicado em 1988 pela Editora Brasiliense, em tradução
indireta de Rogério Silveira Muoio. [N.~da E.]} romance de
literatura fantástica, sobre uma espécie de salamandra inteligente, com grande
capacidade de aprendizagem. Num texto carregado de fina ironia e que põe a nu a
ganância humana, Tchápek mostra que os répteis são subjugados e escravizados
pelos humanos, com o objetivo único de tirar proveito máximo da inteligência
dos animais. No entanto, em decorrência da alta capacidade de aprendizagem, as
salamandras passam a ter vontade própria e reproduzem"-se rapidamente. São
perspicazes a ponto de perceber que são exploradas, fundam sindicatos para
defender os próprios direitos e revoltam"-se, pondo em xeque a posição dominante
do homem na terra. Ao assumir o controle das coisas, contudo, as salamandras
imitam o comportamento humano, numa clara alusão alegórica ao fato de que se
deveria aprender com os erros alheios, e também a evitá"-los, e de que a ganância é
um mal capaz de jungir até os seres mais inteligentes do planeta. Ao
antropomorfizar os personagens do texto, o escritor põe a nu os humanos,
apontando"-lhes os defeitos e retratando"-os como indivíduos que podem tornar"-se
inescrupulosos, gananciosos e aturdidos. Esta obra e, ainda, \textit{Bíla nemoc} (Enfermidade branca)\footnote{ Publicado no Brasil como 
\textit{A doença branca}. Rio de Janeiro: \mbox{Z.~Valverde}, 1942.  [N.~da E.]}  
e \textit{Matka} (Mãe) pertencem à derradeira fase da vida do escritor,
quando se empenhou, mais do que nunca, em combater a crescente influência do
nazi"-fascismo na Europa. Seu nome foi indicado para o Prêmio Nobel de
literatura, mas hoje parece comumente aceita a ideia de que a Academia Sueca
não quis arriscar"-se --- num mundo cada vez mais dominado, à época, pela máquina
de guerra nazista --- concedendo a láurea a um escritor que figurou entre os mais
ardorosos opositores ao regime nefasto que havia tomado o poder em Berlim.

Nas últimas décadas, a literatura tcheca passou a ser mais conhecida por
intermédio das obras de escritores como Milan Kundera (1929--) ou Ivan Klíma
(1931--). No entanto, o humor satírico e corrosivo, o absurdo da condição
humana em decorrência da implantação e atuação de regimes totalitários e  da
ambição pelo poder cultuada pelos homens, ou temas correlatos [todos eles
explorados \textit{ad nauseam}, por exemplo, pelo escritor russo Vladímir Voinóvitch
(1932--), sobretudo no romance \textit{A vida e as aventuras extraordinárias de
soldado Iván Tchomkin} (\russ{\fontfamily{}\selectfont Жизнь} \russ{и}
\russ{необычайные} \russ{приключения} \russ{солдата}
\russ{Ивана} \russ{Чонкина})], sinalizam, de modo incisivo,
que escritores da estatura de Jaroslav Hašek e, sobretudo, Karel Tchápek não
somente  sobreviveram à sua própria época, mas também legaram"-nos uma visão de
mundo ancorada em profundo caráter humanista, e, talvez por isso mesmo e
repetidas vezes, tenham retomado a clássica chave do \textit{ridendo castigat mores}.




