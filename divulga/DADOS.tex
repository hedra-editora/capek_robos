\titulo{A fábrica de robôs}
\autor{Tchápek}  % Apenas sobrenome, se for o caso. Verificar capa.% 
\organizador{Tradução do tcheco}{Vera Machac}   %Conferir se é apenas {Organização}; {Organização e tradução} ou apenas {Tradução}%
\isbn{978-85-7715-161-5}
\preco{21}   % Ex.: 14. Não usar ",00"%
\pag{148}   % Número de páginas
\release{\textsc{A fábrica de robôs} (1920), drama em três atos, pertencente ao ciclo de obras distópicas de Tchápek,  
apresenta um mundo onde o avanço indiscriminado da ciência e da técnica deflagra uma crise sem precedentes 
que ameça a própria humanidade. Um cientista descobre a fórmula capaz de dar vida a máquinas de aparência humana,  
gerando um desequilíbrio radical no modo de produção e tornando
a mão de obra humana obsoleta. Essas ``criaturas'' artificiais, desprovidas de sentimentos e criatividade, 
passam a exercer todas as atividades braçais, com consequências nefastas para os homens.  A palavra ``robô'', 
cujo significado em tcheco é ``servidão; trabalho forçado'', 
e que seria incorporada em quase todas línguas, foi cunhada e usada pela primeira vez nessa peça, encenada a partir de 1921 na Europa, 
com enorme repercussão. Tanto o stalinismo quanto o nazismo ainda estavam sendo gerados
no ano em que a peça foi redigida, mas esta obra constituiu um alerta contra os fundamentalismos ideológicos 
que, logo mais, se abateriam sobre o mundo. 



\noindent\textsc{Karel Tchápek} (Boêmia, 1890--Praga, 1938) é um dos mais celebrados autores tchecos do século \textsc{xx}. Romancista, dramaturgo, 
jornalista e ensaísta, Tchápek foi também contista de talento notável, deixando uma vasta produção. Karel saiu jovem de sua cidade natal, 
situada ao norte da Boêmia. Aos onze anos, foi enviado ao ginásio em Hradec Králové, onde começou a escrever os primeiros textos. 
Em 1904, na cidade de Brno, publicou dois poemas no semanário \textit{Domingo}. Em Praga, estudou filosofia e estética e começou a 
colaborar nos diários mais influentes da capital tcheca com artigos sobre literatura e arte. Também teve uma passagem acadêmica na 
França e Alemanha, onde estudou a cultura germânica. Homem de pensamento livre, tornou"-se o representante 
máximo da cultura democrática de seu país, advertindo os compatriotas e o mundo a respeito do perigo dos fundamentalismos
ideológicos, que varreriam a democracia e a cultura humanística tanto do Velho
Continente quanto de qualquer outro ponto no mapa"-múndi. Tchápek e seu irmão Josef combateram abertamente o nazismo e 
qualquer forma de totalitarismo, e chegaram a ser  declarados inimigos públicos de Berlim. Josef, pintor e escritor, foi enviado em 1939 
para o campo de concentração de Bergen"-Belsen,  de onde nunca retornou.  Tchápek faleceu em decorrência de uma pneumonia,  
três meses após a anexação dos Sudetos pelo regime nazista.

\noindent\textsc{Vera Machac} formou"-se no Colégio Bilíngue associado à Cultura Inglesa, em Praga, estudando a língua tcheca e o inglês.  
Mudou"-se da Tchecoslováquia (atual República Tcheca) para o Brasil em 1951. Cursou tradução e interpretação na Associação Alumni 
e é tradutora juramentada da Junta Comercial de São Paulo (\textsc{jucesp}) desde 2000. 

\noindent\textsc{Aleksandar Jovanović} é doutor em Semiótica e Linguística, 
professor de graduação e pós"-graduação da Faculdade de Educação da 
Universidade de São Paulo e ensaísta e tradutor de línguas 
da Europa Centro"-Oriental.  Dentre outros livros, traduziu, do tcheco, \textit{Histórias apócrifas} 
(Editora 34, 1994), de Karel Tchápek, e \textit{Nem santos nem anjos} (Record, 2006), de Ivan Klíma; do húngaro, 
\textit{História da literatura universal do século xx} (UnB, 1990), de Miklós Szabolcsi, 
e \textit{A exposição das rosas} (Editora 34, 1993), de István Örkény. Do sérvio, 
organizou, prefaciou e traduziu as antologias \textit{Literatura 
iugoslava contemporânea -- Sérvia} (1987) e \textit{Caracol estrelado: poesia sérvia contemporânea da segunda metade do século xx} (2008).


}


