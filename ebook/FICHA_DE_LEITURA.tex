\documentclass[11pt]{hedrabook}
\usepackage[brazilian]{babel}
\usepackage{ucs}
\usepackage[utf8x]{inputenc}
\usepackage[T1]{fontenc}
\usepackage{hedracrop}
\usepackage{hedrabolsolayout}
\usepackage[protrusion=true,expansion]{microtype}
\usepackage{comment,lipsum,footmisc}
\usepackage[minionint,mathlf]{MinionPro}

\makeatletter
  \renewenvironment{frontispiciopage}[2]{
    \thispagestyle{empty}
    \def\@ano{#2}
    \vspace*{\stretch{1}}
    \begin{fancytable}{#1}%
      \putline{\hspace*{36mm}}{}
      \ThirdpageAuthorCathegory               %comentar quando não houver AUTOR
      \putline{T\'itulo}{\textsc{\@title}}}{
      \putline{S\~ao Paulo}{\@ano\vspace{5mm}}
      \bigline{}{}{
        \setlength{\unitlength}{1mm}%
        \hspace{1mm}\begin{picture}(25,25)(0,2.2)%
          %\color{white}
          \put(0, 0){\line(1, 0){2}}
          \put(0, 0){\line(0, 1){2}}
          \put(23, 0){\line(1, 0){2}}
          \put(25, 0){\line(0, 1){2}}
          \put(0, 25){\line(1, 0){2}}
          \put(0, 23){\line(0, 1){2}}
          \put(23, 25){\line(1, 0){2}}
          \put(25, 23){\line(0, 1){2}}
          \put(5.8,13){\makebox(0,0)[l]{\fontencoding{OT1}\fontfamily{ptm}\selectfont\logosize{}hedra}}%
        \end{picture}}
      \bigline{}{%
        \setlength{\unitlength}{\baselineskip}%
        \begin{picture}(0,0)(0,0)%
          \put(0,1){\makebox(0,0)[l]{\interno@traco}}%
          \put(0,2.5){\makebox(0,0)[l]{\interno@traco}}%
          \put(0,3.5){\makebox(0,0)[l]{\interno@traco}}%
          \put(0,5){\makebox(0,0)[l]{\interno@traco}}%
          \put(0,8.7){\makebox(0,0)[l]{\interno@traco}}%
        \end{picture}}{%
        \setlength{\unitlength}{\baselineskip}%
        \hspace{\stretch{1}}\begin{picture}(0,0)(-4,0)%
          \put(-4,1){\makebox(0,0)[l]{\interno@traco}}%
          \put(-4,3){\makebox(0,0)[l]{\interno@traco}}%
          \put(-4,4){\makebox(0,0)[l]{\interno@traco}}%
          \put(-4,10){\makebox(0,0)[l]{\interno@traco}}%
          \put(-4,11){\makebox(0,0)[l]{\interno@traco}}%
          \put(-4,12){\makebox(0,0)[l]{\interno@traco}}%
          \put(-4,16){\makebox(0,0)[l]{\interno@traco}}%
          \put(-4,17){\makebox(0,0)[l]{\interno@traco}}%
          \put(-4,21){\makebox(0,0)[l]{\interno@traco}}%
          \put(-4,23){\makebox(0,0)[l]{\interno@traco}}%
          \put(-4,24){\makebox(0,0)[l]{\interno@traco}}%
        \end{picture}}
    \end{fancytable}}
\makeatother

\AtBeginDocument{%
  \selectlanguage{brazilian}
  \fontsize{11pt}{13.2pt}\selectfont
  \parskip=0pt
  \setlength{\unitlength}{1mm}%
  \setcounter{secnumdepth}{-2}%
  \pagestyle{plain}}


\begin{document}

\author{Karel Tchápek}
\title{A fábrica de robôs}
\begin{frontispiciopage}{4cm}{2013}
  %comentar/modificar linhas abaixo conforme necessário
  \putline{Tradução}{\textsc{Vera Machac}}
%  \putline{Introdução}{\textsc{}}
%  \putline{Organização e tradução}{\textsc{}}
\end{frontispiciopage}


\section{Sobre o autor}

Nascido em 1890 na Boêmia,  Karel Tchápek ocupou o posto de maior escritor
tcheco do início do século XX antes de seu conterrâneo e contemporâneo Franz
Kafka, tendo suas obras traduzidas e elogiadas por críticos, dramaturgos e
escritores na Inglaterra, América e Alemanha.  Além de romances de detetive,
peças satíricas, ensaios críticos, relatos de viagem e artigos, destaca-se na
sua produção diversas obras de cunho fantástico. Neste campo, é pela peça de
1920 \textit{A Fábrica de Robôs} (\textit{R.U.R.: Rosumoví univerzalní
roboti}), que Karel Tchápek é mais lembrado hoje, principalmente entre os
leitores e críticos de Ficção Científica, por ter sido nesta obra que surgiu a
palavra \textit{robô}. Dentre as razões para o seu gradual ostracismo até a
metade do século XX está o fato de que Tchápek escreveu usando o vernáculo de
sua terra, em uma época em que o tcheco nem existia como língua oficial (a
Tchecoslováquia foi criada como país apenas após a Primeira Guerra Mundial).
Kafka, por outro lado, construiu sua carreira usando a mais universal língua
alemã, permitindo assim que seus escritos ultrapassassem as frágeis e instáveis
fronteiras de seu país do período do entre guerras para alcançar o mundo. Este
fato ajudou também a obra de Kafka a escapar da censura do regime comunista que
assumiu o controle do país; o que não ocorreu no caso de Tchápek, cujas
posições políticas sobre democracia e cultura foram repudiadas quando da
ascensão do Comunismo na Tchecoslováquia, banindo toda a produção artística do
autor de \textit{A Fábrica de Robôs} por quatro décadas. Nos últimos anos, no
entanto, os escritos deste editor, escritor, dramaturgo e ensaísta vêm sendo
resgatados por outros escritores da atual República Tcheca, como Milan Kundera
e Ivan Klíma. Karel Tchápek faleceu em Praga, no dia de Natal de 1938 em
decorrência de uma pneumonia.  

\section{Síntese do enredo} 

\textit{A Fábrica de Robôs} é uma peça teatral constituída de uma abertura e
três atos nos quais é descrito como, no passado, a pesquisa com produtos
químicos realizada pelo fisiologista Rossum resultou na descoberta de uma
matéria sintética capaz de simular a vida. Após alguns experimentos fracassados
com animais, este cientista decidiu criar seres humanos.  Anos depois, já como
produtos fabricados pela Robôs Universais Rossum, estas criaturas chamadas de
robôs (do tcheco \textit{robota}, “trabalho forçado”) passaram a ser utilizadas
em larga escala nas indústrias e nas residências para toda sorte de trabalho,
tornando sua presença um elemento indispensável para a própria manutenção da
sociedade.  Gradualmente, porém, os robôs passam a se ressentir da maneira como
são tratados e tomam consciência de sua força frente à humanidade, decidindo se
rebelar contra seus criadores e promover um genocídio de todos os seres
humanos, com exceção de Alquist  ---  engenheiro civil da Robôs Universais
Rossum. O tempo passa e Alquist se mostra incapaz de reproduzir a formula para
a criação de novas criaturas artificiais, apontando para o fim tanto dos robôs
quanto da espécie humana. A esperança de um novo começo, todavia, surge com o
casal de robôs Primus e Helena, que demonstram serem mais evoluídos
emocionalmente em relação aos seus pares robóticos por estarem apaixonados. Na
última cena da peça o velho engenheiro abençoa o casal como os novos Adão e
Eva, que, como tais, irão repovoar o mundo com uma nova espécie. 


\section{BOX A Revolução industrial}

O termo “Revolução Industrial” designa o processo de intenso progresso
científico e tecnológico ocorrido principalmente na Europa entre a metade do
século XVIII ao fim do século XIX e seu impacto na esfera econômica, política,
cultural e artística da sociedade ocidental. Ao primeiro momento da Revolução
Industrial (1760-1860), caracterizado pelo aparecimento das primeiras fábricas,
seguiu-se o segundo momento (1860-1900), marcado pelo surgimento e aplicação de
diversos produtos tecnológicos e científicos, tais como a luz elétrica, o
telefone, a locomotiva elétrica e o motor a diesel. 

\section{Propostas de atividades} 

\paragraph{1.} 

O século XIX como um todo e as primeiras décadas do século XX foram definidos
pelo paradoxal efeito do racionalismo científico em diversas esferas da
sociedade europeia. Se, por exemplo, as inovações tecnológicas e científicas
fomentadas pela Revolução Industrial aumentaram a produção de alimentos no
campo, elas também contribuíram para a extinção dos empregos nas áreas rurais e
o consequente aumento desorganizado das cidades. O mesmo pode ser dito das
ideias e teorias de Charles Darwin, Karl Marx e Sigmund Freud cujos benefícios
para a ciência de suas respectivas áreas também conviveram com a quebra, perda
ou questionamento de valores provocados por estes mesmos estudos. A partir
deste cenário, que encontra equivalência na atualidade com a globalização e os
avanços da informática, pode-se solicitar aos alunos que debatam como Franz
Kafka e Karel Tchápek abordam respectivamente em \textit{A Metamorfose} (1915)
e \textit{A Fábrica de Robôs} (1920), o processo de angústia, ansiedade e
alienação vivenciado pelo homem de virada de século. Sugere-se que em um
primeiro momento o professor divida a classe em dois grupos, ficando cada um
responsável pela leitura e análise de uma obra. Na sequência, a partir da troca
de interpretações sobre cada texto entre os grupos, espera-se que os alunos
percebam os pontos de contatos entre elementos aparentemente tão díspares entre
si quanto insetos e robôs.


\paragraph{2.}

\begin{quote}
  Domin  ---  \ldots{} Mas em até 10 anos a Robôs Universais Rossum produzirá
  tanto trigo, tantos tecidos, tanto de tudo, que digamos: as coisas não terão
  mais valor. Agora cada um pega o que precisa. Não há miséria. Sim, estarão
  desempregados. Mais aí não existirá mais o trabalho. Tudo será feito pelas
  máquinas vivas. As pessoas vão fazer apenas o que gostam. Vão viver apenas para
  se aperfeiçoar. (p. 56)
\end{quote}

Desde a Grécia Antiga, com \textit{A República} (367 A.C.), de Platão,
passando pelo Renascimento com \textit{Utopia} (1516), de Thomas More, até o
século vinte com \textit{A Ilha} (1962), de Aldous Huxley, a idealização de um
lugar onde a paz, a prosperidade e a justiça reinam tem sido um traço
recorrente da história da humanidade. A fala do diretor da fábrica de robôs
Harry Domin em \textit{A Fábrica de Robôs} exemplifica também uma utopia. Tanto
em forma de ideias religiosas e filosóficas quanto de projetos literários e
políticos, a persistência da imaginação utópica sugere que o ser humano anseia
por um mundo perfeito que difira radicalmente do modelo imperfeito vivenciado
diariamente. Mas o que seria então um mundo perfeito? A partir desta questão,
sugere-se que, após a leitura da peça, os alunos elaborem individualmente e sem
identificação de nomes, pequenas descrições do seu modelo de mundo perfeito.
Após esse exercício o professor promove a troca de redações entre os alunos
solicitando que estes observem as possíveis contradições ou exclusões da visão
de utopia por eles analisados. Espera-se ao final da atividade que os alunos
percebam o perigo dos discursos utópicos, uma vez que, por mais idealistas que
estes possam ser, jamais conseguirão contemplar ou evitar a universalidade de
anseios ou medos do ser humano. 


\paragraph{3.}

\textit{A Fábrica de Robôs} divide em lados opostos seres humanos e robôs.
Reproduza este cenário em sala de aula simulando um tribunal com a defesa de
cada lado sendo feita por advogados dos humanos e robôs escolhidos entre os
alunos. Os argumentos de defesa e acusação dos envolvidos devem ser retirados
da fala dos personagens e do próprio desenrolar dos eventos na peça. Qual lado
foi mais convincente? Cabe ao júri, formado pelos alunos, decidir o veredicto.
Como atividade preparatória visando apresentar aos alunos a estrutura de um
tribunal, recomenda-se a exibição do filme \textit{12 Homens e uma Sentença}
(1957).  


\paragraph{4.}

Assim como \textit{A Fábrica de Robôs}, os filmes \textit{Matrix} (1999) e
\textit{Eu, robô} (2004) se alinham ao contexto cultural de virada de século
para discutir a relação homem X sociedade.  Mas quais seriam os pontos em comum
entre esta peça de 1915 e as obras fílmicas produzidas no final do século
passado? Sugere-se aqui a elaboração de uma proposta de produção de texto
individual sobre o tema.


\paragraph{5.}

À luz da ciência de hoje, a descrição da origem e da composição dos robôs da
peça de Karel Tchápek, se encaixam melhor na representação dos clones e
androides da ficção científica moderna. Após chamar a atenção dos alunos para
este fato, o professor pode pedir aos alunos que, em pequenos grupos,
reescrevam a obra de Tchápek usando clones ou androides no lugar de robôs em um
contexto de questões da atualidade, como o conflito entre o local e o
globalizado e a influência das redes sociais nos relacionamentos pessoais,
dentre outros temas. Como conclusão para a atividade, o professor pode propor
aos alunos a representação da peça reescrita. 



\section{Indicações para o professor}

\bigskip

\paragraph{Livros}

\begin{description}\labelsep0ex\parsep0ex
\newcommand{\tit}[1]{\item[\textnormal{\textsc{\MakeTextLowercase{#1}}}]}
\newcommand{\titidem}{\item[\line(1,0){25}]}

\tit{BERMAN}, Marshall. \textit{Tudo que é sólido desmancha no ar:} A aventura da
modernidade. Trad. Carlos Moisés e Ana Ioratti. São Paulo: Companhia
das Letras, 1986.

\tit{BRECHT}, Bertold. \textit{Estudos sobre Teatro}. Trad. Fiama Pais Brandão. Rio
de Janeiro: Nova Fronteira, 1978.

\tit{COELHO}, Teixeira. \textit{O que é utopia}.\textbf{} 9 ed. São Paulo: Editora
Brasiliense, 1992. (Coleção primeiros passos 12).

\tit{FIKER}, Raul. \textit{Ficção científica}: ficção, ciência ou uma épica da
época?\textit{} Porto Alegre: L\&PM, 1985.  (Coleção Universidade Livre).

\tit{HALL}, Stuart. \textit{A identidade cultural na pós-modernidade.} Trad. Tomaz
Tadeu da Silva e Guacira Lopes Louro. 3ed. Rio de Janeiro: DP\&A Editora, 1999.

\tit{HENDERSON}, William Otto. \textit{A revolução industrial}: 1780-1914. Trad.
Maria Ondina.\textit{} São Paulo: Verbo: Ed. Da Universidade de São Paulo,
1979.

\tit{MARX}, Karl, ENGELS, Friedrich. \textit{Manifesto comunista}. Trad. Marcus
Mazzari. São Paulo: Editora Hedra, 2010.

\tit{PAQUOT}, Thierry. \textit{A utopia}:\textit{} ensaio acerca do ideal. Trad.
Maria Helena Kuhner. Rio de Janeiro: Difel, 1999.

\tit{SILVA}, Alexander Meireles. \textit{Literatura Inglesa para Brasileiros}: curso
completo de literatura e cultura inglesa para estudantes brasileiros. 2ed. Rio
de Janeiro: Ciência Moderna, 2005.

\end{description}

\paragraph{Informações sobre Karel Tchápek}

\begin{description}\labelsep0ex\parsep0ex
\newcommand{\tit}[1]{\item[\textnormal{\textsc{\MakeTextLowercase{#1}}}]}
\newcommand{\titidem}{\item[\line(1,0){25}]}

\tit{CLUTE}, John. Karel Čapek. CLUTE, John. (Ed.). \textit{Science Fiction:} The
Illustrated Encyclopedia. London: Dorling Kindersley, 1995, p. 119.

\tit{SUVIN}, Darko. Karel Čapek, Or the Aliens Amongst Us. In: SUVIN, Darko\textit{.
Metamorphoses of Science Fiction}: on the poetics and history of a literary
genre. New York: Yale University Press, 1979. p. 270-284.

\end{description}

\paragraph{Outros livros}

\begin{description}\labelsep0ex\parsep0ex
\newcommand{\tit}[1]{\item[\textnormal{\textsc{\MakeTextLowercase{#1}}}]}
\newcommand{\titidem}{\item[\line(1,0){25}]}

\tit{BRADBURY}, Ray. \textit{Fahrenheit 451}. Trad. Cid Knipel. São Paulo: Biblioteca
Azul, 2012.

\tit{HUXLEY}, Aldous. \textit{Admirável Mundo Novo}. Trad. Vidal de Oliveira e Lino
Vallandro. São Paulo: Globo, 2001. 

\tit{KAFKA}, Franz. \textit{A Metamorfose}. Trad. Celso Donizete Cruz. São
Paulo: Hedra, 2009.

\tit{ORWELL}, GEORGE. \textit{A revolução dos bichos}. Trad. Heitor Aquino Ferreira.
São Paulo: Companhia das Letras, 2007.

\tit{SHELLEY}, Mary. \textit{Frankenstein}. Trad.  Miécio Araújo Jorge Honkins. São
Paulo: L\&PM Editores, 2005.

\end{description}

\paragraph{Filmes}

\begin{description}\labelsep0ex\parsep0ex
\newcommand{\tit}[1]{\item[\textnormal{\textsc{\MakeTextLowercase{#1}}}]}
\newcommand{\titidem}{\item[\line(1,0){25}]}

\tit{}\textit{Metrópolis} (1927), de Fritz Lang

\tit{}\textit{Tempos Modernos} (1936), de Charles Chaplin

\tit{}\textit{12 Homens e uma Sentença} (1957), de Sidney Lumet

\tit{}\textit{Fahrenheit 451} (1966), de François Truffaut

\tit{}\textit{Matrix} (1999), dos irmãos Wachowski

\tit{}\textit{Eu, Robô} (2004), de Alex Proyas

\tit{}\textit{O Preço do Amanhã} (2011), de Andrew Niccol

\end{description}

\paragraph{Na internet}

\begin{description}\labelsep0ex\parsep0ex
\newcommand{\tit}[1]{\item[\textnormal{\textsc{\MakeTextLowercase{#1}}}]}
\newcommand{\titidem}{\item[\line(1,0){25}]}

\tit{}Karel Capek website (Informações em inglês sobre a vida e a obra do artista):
http://capek.misto.cz/\hfil\break english/

\end{description}


\end{document}
